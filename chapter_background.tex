\chapter[Background]{Background and state of the art}
\label{cpt:back}

\section{Introduction}
This chapter will introduce the development and literature review of quantitative finance, stock market investment, and the application of complex networks theory to the financial markets, especially to the stock markets.

\section{Background}
\subsection{Historical review of quantitative finance and stock market investment}
Quantitative finance combines mathematics, statistics, econometrics, machine learning and empirical finance to provide a solid analytical foundation for the analysis of financial issues, especially for the investments in financial markets such as stock, future, option and forex. The theoretical foundations of quantitative finance including efficient market theory, behavioural finance, asset allocation theory and chaos theory.

The revolutionary who pioneered the theory of asset allocation, Markowitz \cite{portfolio} proposed the idea of efficient frontier and the mean-variance model, which established a clear mathematical definition of the two previously vague concepts of risk and return. Markowitz ~\cite{markowitz1956optimization} then presents the properties and formulas for efficient frontier and generalised it by allowing an arbitrary, possibly singular, covariance matrix~\cite{markowitz1959portfolio}. Tobin~\cite{tobin1958liquidity} developed Markowitz's theory and added the concept of risk free assets. Sharp~\cite{equilibrium} and Linter~\cite{diversification} added two key assumptions based on the mean-variance model to enable the portfolio mean-variance valid, forming a capital asset pricing model (CAPM) with the support of economic theories. The CAPM believes that only non-dispersible systemic risks can be compensated off, while non-systematic risks can be eliminated by effectively decentralised investments. Investors could only assume systemic risks through decentralised investment. The systematic risk of a single security or portfolio can be characterised by beta, which represents the extent to which a single security or portfolio is affected by the overall market volatility. And later, Richard Michaud and Robert Michaud \cite{michaud1998asset} addressed the information uncertainty in risk-return estimates by the invention of the Re-sampled Efficient portfolio.

Since the invention of the CAPM, worldwide scholars had been actively conducting empirical tests towards the practicality of the CAPM, while early results show that \textit{beta} is able to explain the return movements of stock prices. However, in the late 1970s, some empirical studies on the CAPM began to show that a large part of the changes in the stock returns cannot be explained by \textit{beta}, as anomalies about market return were increasingly found.

Researchers had proposed models and theories considering the individual stock features to explain such anomalies. For instance, Fama and French \cite{intertemporal, options} proposed a three-factor model based on the inter-temporal capital asset model (ICAPM)  proposed by Merton \cite{riskfactors} and the arbitrage pricing theory (APT) proposed by Cox et al. \cite{anomalies}, which reveals a large part of the cross-section of the average return of stocks that cannot be explained by CAPM, can be explained using firm size, book-to-market equity ratio, and overall market return as regression factors.

Above asset allocation models based on expected return and risk mainly implement the analytical tools of statistics, time series analysis and financial econometrics. The underlying economic ideas were all derived from the rationality of investors. As such methods revolutionised the research of finance, huge lacunae were left striking by the asset allocation theory. For instance, the traditional models are limited in terms of volume. Further, individual investors still hold significantly few stocks in investment portfolios although the importance of diversification are emphasised by these theories. Lastly, expected returns do not seem to vary in the cross-section only due to the risk differentials across stocks.

As Subrahmanyam \cite{subrahmanyam2008behavioural} puts forward, traditional finance theories like asset allocation is limited in explaining issues including (1) the reasons of trading for individuals, (2) the performance of traders, (3) the ways for investors choosing portfolios, and (4) the reasons of returns varying across stocks other than risk. Hence, scholars began to explain financial market with psychology and behavioural economics.

Some typical studies on behavioural finance are related to overconfidence, self-attribution, loss aversion, and short-sale constraints. Daniel et al. \cite{daniel1998investor, daniel2001overconfidence}, Barberis et al. \cite{barberis1998model}, and Hong and Stein \cite{hong1999unified} explain patterns using overconfidence and self-attribution. Overconfidence about private signals causes overreaction and therefore phenomena such as the book/market effect and long-run reversals., while self-attribution maintains overconfidence and allows prices to continue to overreact, creating momentum.

Loss aversion refers to tendency to prefer avoiding losses to acquiring equivalent gains for investors. In terms of modelling such behavioural biases, Barberis et al. \cite{barberis2001prospect} and Barberis and Huang \cite{barberis2001mental} attempt to incorporate the phenomenon of loss aversion into utility functions. Grinblatt and Han \cite{grinblatt2005prospect} argue that loss aversion can also help explain momentum. Scheinkman and Xiong \cite{scheinkman2003overconfidence} analyse the interaction of overconfidence and short-sale constraints. They show that agents with positive information may be tempted to buy overvalued assets because they believe they can sell that asset to agents with even more extreme beliefs. 

From the studies about behavioural finance, it is clear that financial markets are complex systems which cannot be analysed with traditional statistical models such as linear regression. Therefore comparing with the method of psychology or behavioural economy, chaos can be another approach that interpret the "abnormal behaviour" of financial markets in a mathematical way.

The chaos theory for decades was one of the hottest research field in science, while it is rarely applied into finance, timid attempts are made to elucidate the possibilities and outline the limits of its application in finance \cite{sewell2008chaos}. Klioutchnikov et al. \cite{klioutchnikov2017chaos} expect that chaos theory is competitive and may well become a "convenient" theory of the financial market, because traditional finance does not take into account dynamics, while chaos theory is built on the dynamics of the system, which allows the theory to be brought closer to reality. Mandelbrot \cite{mandelbrot2004mis} studies the chaos of stock markets and finds that (1) the changes of stock prices do not strictly follow a normal or Gaussian distribution (2) stock prices show significant autocorrelation, which is incompatible with efficient market theory (3) numerous random factors obscure historical facts. Therefore, the prediction of stock prices is conducted by fractal behaviour to which markets exhibit, and the fractal analysis results show difference from traditional methods. Other methods to predict stock prices based on the considerations of the resumption and wave propagation of events are conducted by Mandelbrot \cite{mandelbrot1983fractal}.
% 简单地介绍一下行为经济学和chaos的发展 然后引出CN

In the last two decades, the notion of complex networks as part of the complexity studies is actively and intensively investigated in the wide filed of networks including financial networks. Therefore, next section will focus on complex networks theory and its applications on financial markets.

%The efficient market theory is mainly based on the assumption of rational expectations and is the theoretical pillar of traditional finance. While behavioural finance is an interdisciplinary subject that combines finance, psychology, behavioural science, and sociology, trying to reveal the irrational behaviours and decision-making patterns in the financial markets. 

\subsection{Theory of complex networks}
A network is defined by a collection of nodes and edges between pairs of nodes. Nodes in large scale brain networks usually represent brain regions \cite{rubinov2010complex}, while edges represent anatomical, functional, or effective connections, depending on the dataset.

Complex networks are the networks that have some or all of the properties from self-organisation, self-similarity, attractors, small-world and scale-free. A large number of complex systems that exist in nature or society can be described by a variety of networks. Since complex networks are the topological basis for the existence of a large number of complex systems, the research on complex networks is believed to help understand the critical problem of "why complex systems are complex".

A typical network consists of a number of nodes and edges which connect these nodes, where nodes are used to represent different individuals in the real system and edges are used to represent the relationships between individuals. In certain cases, a particular relationship is described as an edge between two nodes, otherwise, it is not connected. Two nodes with edge are considered as adjacent among the network.

For instance, the nervous system can be regarded as a network formed by a large number of nerve cells interconnected by nerve fibres~\cite{watts1998collective}. The computer networks can be regarded as a network in which autonomously working computers are connected to each other through communication media such as optical cables, twisted pairs, coaxial cables, etc~\cite{watts1998collective}. In addition, there are more complex networks like electric power networks~\cite{faloutsos1999power}, social networks~\cite{watts1998collective, hofman2017prediction, ebel2002scale}, etc.

The research of complex networks involves the knowledge and theoretical basis of many disciplines due to its cross-disciplinary and complex characteristics, especially for system science, statistical physics, mathematics, computer and information science. Commonly used analytical methods and tools for complex networks research include graph theory, combinatorial mathematics, matrix theory, probability theory, stochastic process, optimisation theory and genetic algorithms. The main research methods for complex networks are based on the theories and methods of graph theory. However, in recent years, many concepts and methods of statistical physics have been successfully applied in the modelling and calculating of complex networks, such as statistical mechanics, self-organisation theory, critical and phase transition theory, seepage theory, etc. \cite{albert2002statistical}, such as the concept of network structure entropy, and its application of quantitative measure of the "order" of complex networks. The models of complex networks are widely used in numerous scientific areas.

It is acknowledged that most practical networks illustrate a structure of community \cite{vazquez2003constrained, newman2003structure}, which refers to groups of nodes that have high inter-density of edges and low intra-density of edges among groups.
It is a matter of common experience that stock market do divide into groups along lines of industry, stock price, revenue, P/E and so forth, and the phenomenon of assortativity discussed in chapter \ref{cpt:result} certainly suggests that this might be the case.
% weighted / unweighted

\vline

Below are some frequently used topological properties and statistical features of complex networks:

\begin{itemize}
	\item Average path length
	\item Clustering coefficient
	\item Degree (strength) and degree (strength) distribution
	\item Centrality
	\item Small-world
	\item Scale-free
\end{itemize}

\vline

Below are some common complex networks models:

\begin{itemize}
	\item Regular network
	\item Random network
	\item Small-world network
	\item Scale-free network
	\item Self-similar network
\end{itemize}

The details of major topology and community features of complex networks will be presented in chapter~\ref{cpt:topology} and the introduction of small-world and random networks will be presented in chapter~\ref{cpt:community}.

\subsection{Complex networks theory applied in Finance}
While traditional stock pricing models still capture limited forms of financial behaviour, according to Johnson et al. \cite{financialcomplex}, the premises of standard financial theory contradict the modern notion of financial markets are complex systems, by which many statistical niceties such as stationarity no longer can be taken for granted. Nevertheless, attempts to assess the prospects of the financial market from the perspective of complex networks theory expand the borders of its application and stimulate the development of new tools and methods of mathematical analysis.

Since the property of small-world and scale-free are respectively revealed by the research upon complex networks by Watts~\cite{watts1998collective} and Barab{\'a}si \cite{barabasi1999emergence}, the application of complex networks has been greatly promoted to each field including finance. Therefore, recent researches have implemented the complex networks theory to reveal the underlying factors of price movements in financial markets. Huang et al.~\cite{chinesenetwork} implemented the threshold method to construct correlation network in A-Share stock market in China and studied the topological properties and topological stability of the stock correlation networks. Their statistical analysis of the degree distribution has revealed the power-law property of financial networks, and the networks display a topological robustness against random node failures, while they are also fragile under intentional attacks. Tabak et al. \cite{tabak2010topological} construct and investigate the stock market networks in Brazil with the minimum spanning tree which is based on the concept of ultrametricity, and implement the correlation matrix for a variety of stocks of distinctive industrial sectors. As a result, they find that stocks tend to cluster by sector. Kau{\^e} Dal’Maso Peron et al. \cite{kaue2012structure} also use the method of stock correlation matrices to analyse the topological properties and dynamic evolution of stock networks in New York Stock Exchange. Additionally they adopt the entropy-related measurement to quantify the robustness of the evolving financial market organisation. As a result, they verify the network topological organisation suffers strong variation during financial instabilities and the networks in such periods become less robust. Namaki et al.~\cite{genuine} utilised Random Matrix Theory (RMT) to specify the biggest eigenvector in the complex networks of price correlations, which reveals that the Tehran Stock Exchange correlation network is scale-free in a specific time interval. Yu~\cite{visibility} studied the evolution of gold price from a network perspective using the visibility network approach and shows that the series of gold price and gold price return are long-term correlated, fractal series with a power-law degree distribution of visibility graph network. Jiang and Zhou \cite{jiang2010complex} construct stock trading networks based on the order flow data of a stock with high-liquidity listed on Shenzhen Stock Exchange in China during a whole year, and find that the trading networks have power-law degree distributions and disassortative architectures. Chopra and Khanna~\cite{intercd} developed a framework which associates the economic input–output model with techniques for understanding interdependencies and interconnectedness in the economy of US, based on complex networks theory. Its topological analysis for two networks suggests that the unweighted network exhibits small world properties, and the weighted network follows a power-law with an exponential cut-off. Boginski et al. \cite{statisticalanalysis} identified cliques and independent groups among stock networks, which  invents a new alternative data mining method to the classification of stocks. Chen et al.~\cite{CHEN2015224} studied the inter-stock and inter-industry effects towards stock returns based on APT and the topological properties of a complex network of correlations. They have found that the average centrality of the top $100$ stocks tracks the GDP growth rate of China, hence, the degrees of connection between stocks in the stock market reflect the development of the real economy to some extent. Another finding is that stocks with smaller market capitalisation tend to be located in more central positions in networks. Brida~\cite{brida2002high} proposed the approach by using symbolic-network model based on coarse graining and symbolisation for calculating the distance of stocks, which is illustrated to be effective to transform stock price time series into complex networks as well, and it does provide an advantageous attempt for analysing time series from the network perspective.

% EIO

\section{Summary}
In this chapter a brief literature review and discussion about quantitative finance, stock market investment and applied complex networks theory have been presented. It can be seen that because of the less regulated price float and tremendously available data, the study of financial market complexity is mainly concentrated on stock data. However, it can also be seen that prevailing complex network approaches to analyse stock markets are almost all about investigating weighted or unweight but undirected networks. To our best knowledge, there is no previous work has attempted to construct a directed network so far.
