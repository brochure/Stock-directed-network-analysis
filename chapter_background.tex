\chapter[Background]{Background and state of the art}
The revolutionary who pioneered the theory of portfolio, Markowitz \cite{portfolio} proposed the mean-variance model and established a clear mathematical definition of the two vague concepts of risk and return. Sharp \cite{equilibrium} and Linter \cite{diversification} added two key assumptions based on the mean-variance model to enable the portfolio mean-variance valid, forming a capital asset pricing model (CAPM) with the support of economic theories. The CAPM believes that only non-dispersible systemic risks can be compensated off, while non-systematic risks can be eliminated by effectively decentralised investments. Investors could only assume systemic risks through decentralised investment. The systematic risk of a single security or portfolio can be characterised by beta, which represents the extent to which a single security or portfolio is affected by the overall market volatility.

Worldwide scholars had been actively conducting empirical tests towards the practicality of the CAPM then, while early results show that beta is able to explain return movements of stocks. However, in the late 1970s, some empirical studies upon the CAPM began to show that a large part of the changes in the stock returns cannot be explained by beta, with increasingly market return anomalies were found.

Researchers had proposed models and theories considering the individual stock features to explain. For instance, Fama and French \cite{riskfactors, anomalies} proposed a three-factor model based on the inter-temporal capital asset model (ICAPM) \cite{intertemporal} and the arbitrage pricing theory (APT) \cite{options}, which reveals a large part of the cross-section of the stocks’ average return that cannot be explained by CAPM, can be explained using firm size, book-to-market equity ratio, and overall market return.

While traditional stock pricing models still capture limited forms of financial behaviour, the premises of standard financial theory contradict the modern notion of financial markets are complex systems \cite{financialcomplex}, by which many statistical niceties such as stationarity no longer can be taken for granted. Recent researches have implemented the network theory to reveal the underlying factors of price movements. Huang et al. \cite{chinesenetwork} implemented the threshold method to build ’s correlation network in China's A-Share stock market and studied the networks topological properties and topological stability. Namaki et al. \cite{genuine} utilised Random Matrix Theory (RMT) to specify the biggest eigenvector in the complex network of price correlations. Yu \cite{visibility} studied the evolution of gold price from a network perspective using the visibility network approach. Chopra and Khanna \cite{intercd} developed a framework which associates the economic input–output (EIO) model with techniques for understanding interdependencies and interconnectedness in the economy of US, based on complex networks theory. Boginski et al. \cite{statisticalanalysis} identified cliques and independent groups among stock networks. Chen et al. \cite{profitable} studied the inter-stock and inter-industry effects towards stock returns based on the topological properties of a complex network of correlations.

In a nutshell, prevailing complex network approaches to analyse stock markets are almost all about investigating weighted or unweight but undirected networks. To the best knowledge of mine, no previous work has attempted to construct a directed network so far.
