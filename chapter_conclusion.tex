\chapter[Conclusions]{Conclusions and future work}
\label{cpt:conclude}
\section{Summary of results}

% Evolve
%This result indicates that a collective behavior is observed between stocks during financial crisis. More specifically, stocks tend to synchronize their price evolution, leading to a high correlation between pair of stock prices, which contributes to the increase in distance between them and, consequently, decrease the network resilience.
This thesis constructed and studied the directed complex networks of US stock market in the year of 2016. From another perspective apart from the methodology presented in this thesis, the construction of the studied directed-weighted network can be also regarded as a network that removed the edges of stock-pairs that are weak in economical transactions and added the directions of edges of stock-pairs by the significance of direct demand or direct requirement from a conventional undirected-weighted stock network.

The essence of introducing the direction of edges is to uplift the difficulty for nodes to connect. Although the new kind of thresholds are based on the fundamental situations of stocks, other than the technical situations of the threshold of correlation coefficient, since the characteristics of topology properties have not changed significantly from conventional stock networks according to the preliminary research in this thesis, we can infer that the new fundamental edge thresholds have similar effect with the technical edge threshold. Furthermore, it does provide an advantageous horizon to elicit more potentially useful information contained in the edge directions. From the new horizon, this thesis is able to analyse on a higher dimensionality — topological property research and community detection with methods for directed networks which utilised the feature of edge directions. The resulting features of power-law and small-world for directed stock complex networks show continuity with the results in undirected stock complex networks researches. The study on community detection suggests "livelihood" and "production" are the most dominant and influential sectors in the stock market and the "finance" sector has extremely strong internal connections. The partitioned communities are highly related with the economical activities among industries and indicate the potential cascading impact from a collapse of a specific firm or sector. The theoretical and practical contributions of aforementioned findings have been discussed.

% This thesis has found the directed stock network is consistent with the undirected-weighted networks about some important topological properties in previous studies, i.e, they are both small-world and scale-free. The study on community detection suggests "livelihood" and "production" are the most dominant and influential sectors in the stock market and the "finance" sector has extremely strong internal connections. This thesis also has found that there is significant negative relationship between standard deviation of stock price return and its corresponding betweenness centrality. 
% theoretical and practical

\section{Future work}
In terms of future work, stock complex networks during a longer range of years can be generated and compared in together, the periods correspond to bull, bear, and stable market can be recognised and categorised for more critical analysis. According to Papadopoulos et al. (2012), the degrees of sub-networks of any power-law networks are not power-law distributed\cite{papadopoulos2012popularity}, hence for a strict demonstration of the power-law distribution for the degrees in stock networks, all stocks in the stock market should be considered in the best-case scenario, but also possible for financial market research. Additionally, more novel and advanced methods for determining the directions of edges to construct the directed complex networks of stock markets are expected to be proposed and implemented.
