\chapter{Introduction}
\section{Motivation}
Financial markets are complex systems, the interconnectedness and interdependencies of industry sectors in the economy are highly inter-coupled with strong correlations with stock price fluctuations, i.e., the price returns of each coupling stocks underlying certain economic link, e.g. two companies that manufacture similar products, or both in one supply chain. Such behaviours can hardly be explained by traditional financial models and theories.

During recent times, weighted but undirected complex network models have been applied to study the correlations of stock prices. Prevailing approach is using companies as nodes, and correlations between each pair of stock price series, return series, or fluctuation patterns as links, e.g., much of the previous researches have proved the represented complex networks of worldwide stock markets are scale-free and small-world~\cite{cnsm, perspective}.

Theoretically, directed complex network for stock market can be achieved therefore more potential information can be produced which is helpful for investment decision and financial market supervision. Conducting Granger causality test between stock pairs is straightforward but not feasible due to the heavy- precondition and time-complexity in programme. Compared to the large order of magnitude of total stock pairs, manually calculated Granger causalities are too less to be used as training samples. Enlightened by the recent work by Jean et al. [3] which uses transfer learning and noisy proxy information per- formed very well at predicting poverties, demonstrating that machine learning techniques is powerful to be applied in a setting with limited training data, so an exploration towards the directed network of stock market is motivated in this project, combines machine learning techniques, transfer learning, individual stock features, and empirical data of Industry Economic Accounts (IEAs) from Bureau of Economic Analysis (BEA) in the US to predict Granger causality of coupling US stocks. Therefore, a directed weighted complex network (DWCN) is constructed by considering companies as nodes, correlations of abnormal stock returns (alpha) as weights of links, and predicted Granger causalities as the in- dications of directions of links.

% hence a new perspective is needed

The goal of this project is to reveal the Granger causality of price return series and utilise them into the topological analysis and visualisation of DWCN as so far no previous work has attempted to construct a directed network about stock price. In addition, suggestions for stock market are provided according to the results and findings.

The outline of this document is as follows. In Section 2, specific objectives and deliverables are listed. In Section 3, relevant previous researches on financial market, complex networks, and machine learning are reviewed. Section 4 de- scribes the analytical methods. Ethics and professional considerations and risk considerations are respectively discussed in Section 5 and 6. Section 7 describes project evaluation approaches and finally, planning Gantt chart is presented in Section 8.

% In the viewpoint of directed networks, the analysis of stock complex network can provide new 

\section{Objectives}
The goal of this project is to construct a directed complex network using economical industrial transaction data and stock price data to depict the US stock market by means of topological properties analysis, community detection and visualisation. Same-sized directed Watt-Strogats small-world network and random networks are generated for the purpose of comparison. This paper will explore whether the conclusions are consistent with the undirected complex network researches.

\section{Proposed methodology}
% summary and put a diagram of the part of my thesis
\subsection{Construction of the stock network}
\subsubsection{EIO (Economic Input-Output) table}


\subsection{Logarithmic return of stock prices}

\subsubsection{Correlation coefficient}


\subsection{Directed Watts–Strogatz small-world network}



\subsection{Directed random network}

\subsection{Topological properties of directed networks}
\label{sec:aim}


\section{Summary of results}

\section{Outline of the thesis}

