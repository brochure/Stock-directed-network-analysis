\chapter{Other resource}
\label{cpt:other}
\section{Introduction}
\subsection{Motivation}
Conducting Granger causality test between stock pairs is straightforward but not feasible due to the heavy-precondition and time-complexity in programme. Compared to the large order of magnitude of total stock pairs, manually calculated Granger causalities are too less to be used as training samples. Enlightened by the recent work by Jean et al. [3] which uses transfer learning and noisy proxy information performed very well at predicting poverties, demonstrating that machine learning techniques is powerful to be applied in a setting with limited training data, so an exploration towards the directed network of stock market is motivated in this project, combines machine learning techniques, transfer learning, individual stock features, and empirical data of Industry Economic Accounts (IEAs) from Bureau of Economic Analysis (BEA) in the US to predict Granger causality of coupling US stocks. Therefore, a directed weighted complex network is constructed by considering companies as nodes, correlations of abnormal stock returns (alpha) as weights of links, and predicted Granger causalities as the indications of directions of links.

\subsection{Objectives}
The goal of the attempt in this chapter is to reveal the Granger causality of price return series and utilise them into the topological analysis and visualisation of directed complex networks as so far no previous work has attempted to construct a directed network about stock markets. In addition, suggestions for stock market can be provided according to the results and findings.

\subsection{Methodology}

% theoretical and practical

\section{Results}
In terms of future work, stock complex networks during a longer range of years can be generated and compared in together, the periods correspond to bull, bear, and stable market can be recognised and analysed separately and accordingly. New methods for determining the directions of edges to generate directed complex networks are expected to be proposed.

\section{Considerations}
