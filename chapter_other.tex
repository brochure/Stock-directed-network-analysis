\chapter{Other results}
\label{cpt:other}
\section{Introduction}
Due to the complexity of financial market and the interconnectedness and interdependencies of industrial sectors in the economy, the price returns of each coupling stocks might have certain underlying economic link. Such behaviours can hardly be explained by traditional financial models and theories. This chapter attempts to combine machine learning techniques, individual stock features, and empirical data of Economic Input-Output (EIO) from Bureau of Economic Analysis (BEA) in the US to predict Granger causality of coupling US stocks. Limited Granger causalities are calculated as a small sample set compared to the target date set. Therefore a directed complex network is expected to be constructed. The generated directed stock network is planned to be analysed about its topological properties, stability and effects on individual stocks and industries. Suggestions towards financial market investment are expected to be provided based on the results in this study.

Financial data of listed companies and fundamental economical data are both available in each stock market and government websites. Efforts have been taken upon the researches such as the work of Patel et al. \cite{patel2015predicting}, which applied machine learning techniques to predict stock price movement, but most of them use correlations between stock price or return series, such measures are unable to provide direction information for building a directed graph of stock market. Granger causality test is one suitable measure but the computation is overwhelmingly complex so that no researchers have ever implemented this.

This chapter has probed into the feasibility of applying machine learning techniques helping to predict Granger causality based on samples of Granger causalities that have been manually calculated. Here the word “predict” means estimation of some property that is not directly observed, rather than its common meaning of inferring something about the future. Unfortunately, over 3,000 listed companies yielding couples many orders of magnitude larger than the amount of sample data for human-beings can ever calculate. The scarcity of training data on these outcomes makes the application of machine learning techniques challenging.

This project overcome this challenge through a multi-step “transfer learning” approach \cite{pan2010survey}, whereby a noisy but easily obtained proxy for sectoral association, the correlations of stock pairs, and fundamental indicators of listed companies are used to train a deep learning model. The model is then used to estimate Granger causalities based on very limited samples through a transferring process.

\subsection{Motivation} % DONE
Conducting Granger causality test between the price return series of all stock pairs is straightforward while not feasible due to the heavy-precondition for Granger causality test in time series analysis and the high time-complexity in computer programme, as this is an NP-hard problem. Hence, predicting Granger causality of the price return series of stock pairs using a training sample which contains manually calculated Granger causalities. However, compared to the large order of magnitude of total stock pairs, the realistic number of manually calculated Granger causalities are too less to be regarded as training samples. Enlightened by the recent work of Jean et al. (2016) which implements transfer learning and noisy proxy information performed unexpectedly well at predicting poverties, demonstrating that machine learning techniques are powerful to be applied in a setting with limited training data \cite{jean2016combining}, therefore an exploration towards the directed network of stock market is motivated in this thesis, combines machine learning techniques, transfer learning, individual stock features, and empirical data of Economic Input-Output (EIO) from Bureau of Economic Analysis (BEA) in the US to predict Granger causalities of rest coupling US stocks. Therefore, the directed stock network is able to be constructed by considering predicted Granger causalities as the indicators of directions of edges.

\subsection{Objectives} % deliverables list
The goal of the attempt in this chapter is to predict and reveal the Granger causality of price return series with machine learning techniques and utilise them into the construction and topological analysis of the directed stock network as so far no previous work has ever been attempted to construct a directed network of stock markets. While in hindsight this is not a successful attempt, conclusion will be drew according to the results of prediction.

\section{Methodology}

\begin{figure}
	\begin{center}
		\includegraphics[width=11cm]{methodology_diagram_dnn}
	\end{center}
	\caption{The methodology diagram of machine learning pathway}
	\label{fig:methodology_diagram_dnn}
\end{figure}

\subsection{Technical challenges in the pre-processing for network construction}
A preliminary correlation matrix is generated for the co-integration test of two price return time series. Bivariate Granger causality test is heavy on the precondition of time series analysis and its time-complexity in computer programme is polynomial, as the amount of all coupling stock pair is $n(n-1)/2$, while $n$ is the number of stocks. There are over 3,000 companies listing in NASDAQ, hence there should be millions of times for Granger causality tests and pre-process time series analyses to run in the computer programmes in order to construct the directed stock network.

However, in this chapter, machine learning techniques are applied to predict the precedence relations, i.e., predicted Granger causalities of every possible US stock pairs, based on a limited amount of actual Granger causalities calculated as training set. In addition to the fundamental indicators such as market capitalisation, P/B ratio, P/E ratio, etc., public empirical data of EIO are also implemented into the entire machine learning prediction process. Stocks are divided into industrial groups according to the summary level defined in the BEA. Transfer learning technique is applied as well and will be introduced in section \ref{sbs:transfer}. In addition, learning performances of each models are compared based on their predicting performances according to the ROC analysis.

\subsection{Granger causality test}
Granger causality test \cite{granger1969investigating} provides an asymmetrical measure for testing precedence relationship between two time series. The leitmotiv inside is that a time series can be described and analysed through a time-delayed auto-regressive model. Granger causality test tests whether the difference of a prediction to the time series from another time series through a multi-variate auto-regressive model is able to improve the prediction of the current behaviour of the time series, as the following forms illustrates:

\begin{eqnarray}
&&	x_t=\sum_{i=1}^{\infty}a_ix_{t-i}+c+\varepsilon_{t}\\
&&x_t=\sum_{i=1}^{\infty}a_ix_{t-i}+\sum_{j=1}^{\infty}b_jy_{t-j}+c'+\varepsilon_{t}'
\end{eqnarray}

Calculate the f-statistic using the following equation, the Granger causality is not significant if f-statistic is greater than the f-value:
\begin{eqnarray}
&&	F=\frac{(ESS_R-ESS_{UR})/q}{ESS_{UR}/(n-k)}
\end{eqnarray}

\subsection{Transfer learning and ANN model}
\label{sbs:transfer}
The most common and basic assumptions for the sampling of machine learning are: (1) the training sample and the test sample are both independent and identically distributed; (2) there must be enough available training samples \cite{rasmussen2004gaussian}. However, as aforementioned reasons in this chapter, the application of machine learning in stock market to predict Granger causalities is hard to satisfy these assumptions.

Transfer learning with existing knowledge to solve only one small target area labelled sample data even without data of learning problems, fundamentally relaxes the basic assumptions of conventional machine learning. Transfer learning can migrate the models which are applicable to big data sets to small data sets, identify the commonality of different problems, and then transfer the generalised model on customised data sets to achieve customised transfer objectives.

The initial set of parameters of ANN model that is trained in this thesis has 3 layers of neurons and 10 nodes in each layers. The activation function for each nodes is rectified linear unit (ReLU) \cite{hahnloser2000digital}, as the below expression shows:
\begin{eqnarray}
f\left(x\right)=x^+=max\left(0,x\right)
\end{eqnarray}

During the training process, Adagrad optimiser \cite{duchi2011adaptive} with the learning rate of 0.01 are applied.

As the methodology diagram \ref{fig:methodology_diagram_dnn} illustrates, since only hundreds of times of Granger causality tests and the pre-works such as co-integration tests can be conducted towards randomly selected stock pairs within a reasonable period of time, the ANN model cannot be directly trained to predict Granger causalities of the rest stock pairs. According to the basic assumptions, there must be enough training samples.

For solving the scarcity of sample, this thesis applies the transfer learning approach and builds the source set for pre-training with EIO data, co-integration test data, stock fundamental data and correlation coefficient data which is regarded as the proxy for Granger causalities. The parameters of ANN model will be fine-tuned during the following optimisation stage, until an satisfying prediction accuracy for correlation coefficients is achieved.

In the first step of the transfer learning precess, the 3-layer ANN model previously trained is fine-tuned through the source set to predict Granger causalities given the corresponding EIO data and a small set of manually calculated Granger causality results. We treat this step of the transfer learning approach as a classification problem, of which the three classes of stock network directions correspond to forward, backward, and neutral. Given a stock pair $[A,B]$, a forward class corresponding to the price return fluctuation preceding direction from stock $A$ to $B$ and vice versa, while a neutral class indicates that there is no significant price return fluctuation preceding influence between stock $A$ and $B$.

% theoretical and practical

\section{Results}
\begin{figure}
	\centering
	\subfloat[Accuracy]{
		\label{subfig:accuracy}
		\includegraphics[width=12.5cm]{accuracy} }
	
	\subfloat[Average loss]{
		\label{subfig:average_loss}
		\includegraphics[width=12.5cm]{average_loss} }
	
		\subfloat[Loss]{
		\label{subfig:loss}
		\includegraphics[width=12.5cm]{loss} }
	
	\caption{Performance of prediction. Graphs are generated by \textit{TensorBoard} \cite{tensorflow2015-whitepaper}.}
	\label{fig:prediction}
\end{figure}

Figure \ref{subfig:accuracy} illustrates the results of accuracy under the circumstance of the initial ANN model, i.e., with 3 neuron layers and 10 nodes in each layer. Since the accuracy of prediction towards the correlation coefficients in training set and test set are around 0.373 and 0.337 respectively, the performance are even worth than random guess which should be 0.333 as there are three distinctive classes. The loss is calculated by using softmax cross entropy \cite{tensorflow2015-whitepaper}, and the average loss is calculated by loss divided by the batch size which is 50 set in the computer programme. According to the figure \ref{subfig:average_loss} and figure \ref{subfig:loss}, as more sample data input to the ANN model through iterations, the loss and average loss gradually decrease until the asymptotic lines of 50 and 1, which indicate the random-guess prediction. Similar results applied to all parameter combinations through the model optimisation approach of grid search.

\section{Conclusion}
The machine learning pathway of the edge direction approach has been briefly presented. As a result, the underperformed outcomes in the first step of transfer learning indicate that the subsequent processes could not be performed. This thesis considers that the predictive patterns to the Granger causality beneath the data of stock price return correlation coefficients, stock fundamental and economical transactions are insignificant and therefore a new pathway of using direct demand and direct requirement from EIO data is explored as the main content of this thesis presents.
