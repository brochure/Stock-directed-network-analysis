\chapter[Network Topological properties]{Topological properties of directed complex networks}
\section{Degree centrality and strength centrality}
The degree of a node $k$ represents the number of its neighbours. In directed network, out-degree $k_{out}$ is the number of edges which start from the given node and end at others, while in-degree $k_{in}$ is the number of edges which end at the given node and start from others. Thus, there is relationship between $k_{in}$ and $k_{out}$:
\begin{eqnarray}
k=k_{in}+k_{out}.
\end{eqnarray}

As one of the most widespread measures to calculate network centrality, degree centrality of a node can be described as the number of direct links that relate to a specific node \cite{freeman}. In terms of the directed stock price return network, this paper mainly focuses on the out-degree analysis on the nodes. Moreover, the strength centrality has generally been accumulated to the sum of weights of out-degrees to form the weighted networks. The equation of this measure is shown as bellow:
\begin{eqnarray}
&&C_D^W(i)=\sum_{j}^{N}w_{ij}
\end{eqnarray}

where $W$ represents the matrix of weighed adjacencies, and $w_{ij}$ represents the weight of the link between node $i$ and $j$.

\section{Degree distribution and strength distribution}
The degree distribution of stock price return network $p(k)$ can be defined as:
\begin{eqnarray}
p_d(k)=\frac{N_k}{N},
\end{eqnarray}

while $N_k$ represents the number of nodes whose out-degree value is k. The distribution of strength has a similar definition:
\begin{eqnarray}
p_s(w)=\frac{N_w}{N},
\end{eqnarray}

while $N_w$ represents the number of nodes whose strength value is w.

\section{Average shortest path length}
The average shortest path length of a directed network $G$ is defined as the following equation:
\begin{eqnarray}
l_G=\frac{1}{n(n-1)}\sum_{i,j\in V}{d(i,j)}
\end{eqnarray}

where $V$ is the set of nodes of $G$.

\section{Betweenness centrality}

Other than strength, betweenness centrality~\cite{freeman1977set} can be used to determine the critical nodes among the entire network and to recognise the most associated firms in the chosen stock market. When it comes to weighted networks, betweenness centrality of a node is the sum of the weights in the fraction of all-pairs shortest paths that pass through this node, which can be described as the following equation:
\begin{eqnarray}
&&C_B(v)=\sum_{s,t \in V}\frac{\sigma(s,t|v)}{\sigma(s,t)}
\end{eqnarray}

where $V$ is the set of nodes, $\sigma(s,t)$ is the sum of weights of all-pairs shortest $(s,t)$-paths, and $\sigma(s,t|v)$ is the sum of weights of those paths passing through some node $v$ other than $s,t$. If $s=t$, $\sigma(s,t)=1$, and if $v\in s,t$, $\sigma(s,t|v)=0$.

\section{Clustering coefficient} %%*改写*
Clustering coefficient is a measure of the degree to which nodes in a network tend to cluster together. Concerning the clustering coefficient of the complex networks, it is defined as:
\begin{eqnarray}
C_i=\frac{2E_i}{(k_i(k_i-1))},
\end{eqnarray}

where $k_i$ is the degree of a given node $v_i$, $E_i$ is the real existing edges among the nearest neighbour nodes of the given node $v_i$, and $k_i(k_i-1)/2$ means the maximum possible edges existing between its nearest neighbours of the node vi. Besides, the clustering coefficient of a node accounts for the extent to which the transmission relationship between the given node and its neighbours also exists between its neighbours, and the clustering coefficient may be given by:
\begin{eqnarray}
C=\frac{3\times number\ of\ triangles\ in\ the\ networks}{number\ of\ connected\ triples\ of\ nodes}.
\end{eqnarray}

This measure gives an indication of the clustering in the whole network, and can be applied to both undirected and directed networks.

\section{Efficiency}
Network efficiency measures how efficient for information being conducted and exchanged in the network, which can help to determine whether the objective network shows small-world property. There are global and local efficiencies that on the different scale sizes~\cite{latora2001efficient}.

\subsection{Global efficiency}
Global efficiency quantifies the conduction and exchange of information through out the entire network. The global efficiency of network \textbf{G} is defined as:
\begin{eqnarray}
&&E_{glob}(\textbf{G})=\frac{\sum_{i \neq j \in \textbf{G}} \epsilon_{ij}}{N(N-1)}=\frac{1}{N(N-1)}\sum_{i \neq j \in \textbf{G}} \frac{1}{d_{ij}}
\end{eqnarray}

\subsection{Local efficiency}
The local efficiency evaluates the resistance of a network towards node \textit{i} and quantifies the conduction and exchange of information among its neighbours. The local efficiency of node \textit{i} in network \textbf{G} is defined as:
\begin{eqnarray}
&&E_{loc}(G, i)=\frac{1}{N} \sum_{i \in \textbf{G}} E_{glob}(\textbf{G}_i)
\end{eqnarray}

\section{Assortativity and Degree Correlations}

The phenomenon of assortative~\cite{newman2002assortative} mixing can be quantified by means of an assortative
coefficient. Let $E_{ij}$ be the number of edges in the network that connect
a vertex of type $i$ to one of type $j$, with $i, j=1, . . . , n$, then similar in spirit to the
adjacency matrix for vertices, these edges can be represented in the form of an edge
incidence matrix $\mathbf{E}$, with elements $E_{ij}$. A normalized mixing matrix is defined as follows:
\begin{eqnarray}
&&\mathbf{e}=\frac{\mathbf{E}}{\|\mathbf{E}\|},
\end{eqnarray}

where $\|\mathbf{E}\|$ refers to the sum of the elements of the matrix $\mathbf{E}$. The entries $e_{ij}$ in the normalized matrix represent the fraction of edges that connect vertices of types $i$ and $j$, and satisfies the normalization condition,

\begin{eqnarray}
&&\sum_{ij}e_{ij}=1.
\end{eqnarray}

The assortativity coefficient $r$ is then defined thus,
\begin{eqnarray}
&&r=\frac{Tr(\mathbf{e})-\|\mathbf{e}\|^2}{1-\|\mathbf{e}\|^2},\label{con:degreecorr}
\end{eqnarray}

where $Tr(\mathbf{e})$ is the standard matrix trace—the sum of the diagonal elements $e_{ii}$. The value of the coefficient $r$ lies in the range $-1\leq{r}\leq{1}$, where 1 represents a perfectly assortative network, 0 a randomly mixed one and -1 a perfectly disassortative network.

Since the degree is an important topological measure, degree correlations assume a significant amount of relevance as they can give rise to complicated network structural effects. The degree correlation can be computed using Eqn.~\ref{con:degreecorr}, where the elements $e_{ij}$ represent the fraction of edges that connect a vertex of degree $i$ to that with degree $j$.

\section{Modularity}
Modularity stands for the difference between fraction of links that fall within communities and the expected fraction if links are randomly distributed~\cite{newman2004finding}. This project introduces modularity as a measure to evaluate the connection strength between node pair within a group. Regarding to the industry where the stocks belong to, these stocks are divided into different groups hence modularity is used to measure the closeness of intra- and inter-group.

Two groups are combined to generate the modularity value while computing the closeness of two groups, as formula below shows:
\begin{eqnarray}
&&Q=\frac{1}{2m}\sum_{j}\left[w_{ij}-\frac{k_ik_j}{2m}\right]\delta\left(c_i,c_j\right)
\end{eqnarray}

where $c_i$ is the community to which node $i$ is assigned, and $k_i$ represents the degree of node $i$. The $\delta$-function $\delta(u,v)$ is 1 if $u=v$ and 0 otherwise and $m=0.5\sum_{ij}w_{ij}$ is the sum of weights in the whole network.
